\documentclass[12pt]{report}
\usepackage[a4paper, margin=1in]{geometry}
\usepackage{setspace}
\usepackage{titling}
\usepackage{graphicx}
\usepackage{lmodern}
\usepackage{subcaption}
\usepackage{amsmath}

\begin{document}

% First Page - Centered
\begin{titlepage}
    \centering
    \vspace*{1cm}

    \includegraphics[width=0.3\textwidth]{media/dukeLogo.png}\\[3cm]

    \begin{spacing}{2} % You can adjust this value (e.g., 1.3, 1.5)
    \centering
    \MakeUppercase{\Huge \textbf{Forecasting Stock Market Trends using Learning-Based Time-Series Modeling Techniques}}
\end{spacing}

\vspace{10.5ex}

    {\Large \textbf{Authors: }\\[3ex]
    Mohammad Afrazi\\[2ex]
    Patrick Lo}\\[5cm]
\end{titlepage}

% Second Page - Left-Aligned and Bottom-Aligned
\newpage

% \vspace{12pt}

% SECTION 1: PROBLEM DESCRIPTION
\section*{Problem Description}

Predicting stock market movements is highly challenging due to the market's volatility, non-linearity, and chaotic dynamics \cite{fischer2018deep}. Traditional time-series models such as ARIMA struggle to capture complex temporal dependencies in financial data, limiting forecasting accuracy and decision-making in finance \cite{khashei2011novel}. 

We argue and hypothesize that learning-based time-series models - specifically Long Short-Term Memory (LSTM) networks, Temporal Convolutional Networks (TCNs), and Transformer- based models - can better model long-term dependencies in stock price data. This project will develop and compare these models for forecasting the daily closing price of the \textbf{S\&P~500} index using historical price. Via evaluating LSTM, TCN, and Transformer approaches on a recent dataset of U.S. market data, we aim to advance financial forecasting methods and provide more reliable tools for investment and risk management \cite{nelson2017stock}.


% SECTION 2: DATA DESCRIPTION
\section*{Data Description}

To address our research question, we utilize the "US Stock Market Dataset" from Kaggle \cite{saketk511}, which contains over five years of daily stock and commodity data for a diverse set of U.S.-listed companies and assets. Each record includes the following features: \textit{No, Date, Natural\_Gas, Natural\_Gas\_Vol., Crude\_oil, Crude\_oil\_Vol., Copper, Copper\_Vol., Bitcoin, Bitcoin\_Vol., Ethereum, Ethereum\_Vol., S\&P\_500, Nasdaq\_100\_Price, Nasdaq\_100\_Vol., Apple, Apple\_Vol., Tesla, Tesla\_Vol., Microsoft, Microsoft\_Vol., Silver, Silver\_Vol., Google, Google\_Vol., Nvidia, Nvidia\_Vol., Berkshire, Berkshire\_Vol., Netflix, Netflix\_Vol., Amazon, Amazon\_Vol., Meta, Meta\_Vol., Gold, Gold\_Vol.}

To mitigate the effects of major scale differences across time periods, we restrict our analysis to data from January 2, 2020, to November 29, 2022, yielding a total of 718 daily observations. Figures~\ref{fig:asset_price} and~\ref{fig:data_preview} provide visual summaries of the dataset, illustrating overall asset price dynamics and a sample of the input features used in our analysis.

\begin{figure}[htbp]
    \centering
    \begin{subfigure}[b]{0.48\textwidth}
        \centering
        \includegraphics[width=\textwidth]{media/asset_price.pdf}
        \caption{Asset price trends.}
        \label{fig:asset_price}
    \end{subfigure}
    \hfill
    \begin{subfigure}[b]{0.48\textwidth}
        \centering
        \includegraphics[width=\textwidth]{media/data_preview.pdf}
        \caption{Data preview.}
        \label{fig:data_preview}
    \end{subfigure}
    \caption{Visualization of the dataset showing asset price dynamics and a sample of the input data.}
    \label{fig:data_visuals}
\end{figure}


\section*{Preprocessing}
\subsubsection*{Log Return}
To make the data more stationary and suitable for modelling, we first compute the \textbf{log returns} of each asset. This is to normalize price movements and stabilize variance across different scales. The log return for time $k$ is calculated as  
\begin{align*}
    r_k = \ln\left(\frac{P_k}{P_{k-1}}\right),
\end{align*}
\noindent where $P_k$ represents the asset price at time $t$. It also helps mitigate the effects of large price discrepancies between assets, ensuring that model training focuses on relative changes rather than absolute price levels. Compared to figure \ref{fig:asset_price}, below shows the log-return transformed of all asset prices.
\begin{figure}[htbp]
    \centering
    \includegraphics[width=0.8\textwidth]{media/all_asset_log_returns.pdf}
    \caption{Log-return transformed asset price trends.}
    \label{fig:all_asset_log_returns}
\end{figure}

\subsubsection*{Dimension Reduction}
Here we perform \textbf{lag-principal component analysis (Lag-PCA)} to identify the most informative features in the dataset and reduce dimensionality. Since we are dealing with time-series data, we create a lagged version by sliding a window of size $l = 70$ over the time series. For each data chunk, we perform PCA independently to capture local temporal patterns. We then apply a \textbf{majority voting} scheme across all chunks to determine the most significant principal components ($n = 10$) to retain for model training. The majority voting is updated via the following heuristic:
\begin{align*}
    V_i \leftarrow V_i + n * \frac{1}{\text{rank}_i}.
\end{align*}
Here, $\text{rank}_i$ denotes the rank of the $i^{\text{th}}$ component within a chunk, and $n$ is the number of components considered. The components with the highest aggregated scores across all windows are selected as the final features for model training. Figure \ref{fig:pca_feature_importance} shows the final result of the Lag-PCA.
\begin{figure}[htbp]
    \centering
    \includegraphics[width=0.8\textwidth]{media/pca_feature_importance.pdf}
    \caption{Lag-PCA feature importance.}
    \label{fig:pca_feature_importance}
\end{figure}

\section*{Methods}
In this project, we implement and compare three learning-based time-series models: Long Short-Term Memory (LSTM) networks, Temporal Convolutional Networks (TCNs), and Transformer-based models. 

\subsubsection*{LSTM}
We implement the LSTM model using PyTorch. The architecture consists of two LSTM layers followed by two fully connected layers. The model is trained using the Adam optimizer with a learning rate of 0.001 and Mean Squared Error (MSE) as the loss function. We employ early stopping based on validation loss to prevent overfitting.

Below shows the model architecture:
\textbf{include figure here}

% \subsubsection*{TCN}
% We implement the TCN model using PyTorch. The architecture consists of several causal convolutional layers with dilated convolutions to capture long-range dependencies. The model is trained using the Adam optimizer with a learning rate of 0.001 and Mean Squared Error (MSE) as the loss function. Similar to the LSTM model, we use early stopping based on validation loss to prevent overfitting.

\section*{Preliminary Results}
Here we include the preliminary results of the LSTM model. The training and validation loss curves are shown in Figure \ref{fig:lstm_loss_curve}. We observe that the training loss decreases steadily, while the validation loss starts to increase after a certain number of epochs, indicating potential overfitting.

\section*{Current Conclusions, Unexpected Challenges, Next Steps}
\begin{itemize}
    \item Successfully implemented the \textbf{LSTM} model for time-series prediction.
    \item Observed that the test loss is larger than the training loss, indicating possible \textbf{overfitting}. To mitigate this, we plan to gather and incorporate more data.
    \item With sufficient data, we intend to implement and compare the performance of \textbf{Temporal Convolutional Networks (TCN)} and \textbf{Transformer-based models} against the LSTM.
    \item Additionally, we will compare the results with traditional \textbf{regression models} to evaluate relative performance.
    \item We plan to investigate the effect of dimensionality reduction by examining whether reducing the number of \textbf{PCA components} significantly degrades model performance.
    \item Finally, we aim to train a model using the \textbf{raw price data} (without log-return transformation) and compare its results with the log-return–based model.
\end{itemize}



% \subsubsection*{TCN}

\vspace{3mm}

\textbf{Work Breakdown:}
\begin{itemize}
    \item \textbf{Mohammad Afrazi:} Data preprocessing, feature engineering, writing, and model training/evaluation.
    \item \textbf{Patrick Lo:} Data preprocessing, feature engineering, writing, and model training/evaluation.
\end{itemize}


\renewcommand{\bibname}{References}

\bibliographystyle{plain}
\bibliography{Reference}

\end{document}

