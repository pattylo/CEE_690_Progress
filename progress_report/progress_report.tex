\documentclass[12pt]{report}
\usepackage[a4paper, margin=1in]{geometry}
\usepackage{setspace}
\usepackage{titling}
\usepackage{graphicx}
\usepackage{lmodern}
\usepackage{subcaption}
\usepackage{amsmath}

\begin{document}

% First Page - Centered
\begin{titlepage}
    \centering
    \vspace*{1cm}

    \includegraphics[width=0.3\textwidth]{media/dukeLogo.png}\\[3cm]

    \begin{spacing}{2} % You can adjust this value (e.g., 1.3, 1.5)
    \centering
    \MakeUppercase{\Huge \textbf{Forecasting Stock Market Trends using Learning-Based Time-Series Modeling Techniques}}
\end{spacing}

\vspace{10.5ex}

    {\Large \textbf{Authors: }\\[3ex]
    Mohammad Afrazi\\[2ex]
    Patrick Lo}\\[5cm]
\end{titlepage}

% Second Page - Left-Aligned and Bottom-Aligned
\newpage

% \vspace{12pt}

% SECTION 1: PROBLEM DESCRIPTION
\section*{Problem Description}

Predicting stock market movements is highly challenging due to the market's volatility, non-linearity, and chaotic dynamics \cite{fischer2018deep}. Traditional time-series models such as ARIMA struggle to capture complex temporal dependencies in financial data, limiting forecasting accuracy and decision-making in finance \cite{khashei2011novel}. 

We argue and hypothesize that learning-based time-series models - specifically Long Short-Term Memory (LSTM) networks, Temporal Convolutional Networks (TCNs), and Transformer- based models - can better model long-term dependencies in stock price data. This project will develop and compare these models for forecasting the daily closing price of the \textbf{S\&P~500} index using historical price. Via evaluating LSTM, TCN, and Transformer approaches on a recent dataset of U.S. market data, we aim to advance financial forecasting methods and provide more reliable tools for investment and risk management \cite{nelson2017stock}.


% SECTION 2: DATA DESCRIPTION
\section*{Data Description}

% Old version  
% To address our research question, we utilize the "US Stock Market Dataset" from Kaggle \cite{saketk511}, which contains over five years of daily stock and commodity data for a diverse set of U.S.-listed companies and assets. Each record includes the following features: \textit{No, Date, Natural\_Gas, Natural\_Gas\_Vol., Crude\_oil, Crude\_oil\_Vol., Copper, Copper\_Vol., Bitcoin, Bitcoin\_Vol., Ethereum, Ethereum\_Vol., S\&P\_500, Nasdaq\_100\_Price, Nasdaq\_100\_Vol., Apple, Apple\_Vol., Tesla, Tesla\_Vol., Microsoft, Microsoft\_Vol., Silver, Silver\_Vol., Google, Google\_Vol., Nvidia, Nvidia\_Vol., Berkshire, Berkshire\_Vol., Netflix, Netflix\_Vol., Amazon, Amazon\_Vol., Meta, Meta\_Vol., Gold, Gold\_Vol.}
%%

To address our research question, we utilize the "US Stock Market Dataset" from Kaggle \cite{saketk511}. This dataset contains over five years of daily data, comprising 36 features. These features include price and volume data for major stock indices (S\&P~500, Nasdaq 100), key individual stocks (e.g., Apple, Tesla, Nvidia), major commodities (e.g., Crude Oil, Gold, Silver), and leading cryptocurrencies (e.g., Bitcoin, Ethereum). Our primary objective is to forecast the S\&P~500 closing price, treating the remaining asset data as predictive features.


To mitigate the effects of major scale differences across time periods, we restrict our analysis to data from January 2, 2020, to November 29, 2022, yielding a total of 718 daily observations. Figures~\ref{fig:asset_price} and~\ref{fig:data_preview} provide visual summaries of the dataset, illustrating overall asset price dynamics and a sample of the input features used in our analysis.

\begin{figure}[htbp]
    \centering
    \begin{subfigure}[b]{0.45\textwidth}
        \centering
        \includegraphics[width=\textwidth]{media/asset_price.pdf}
        \caption{Asset trends: Bitcoin, S\&P 500, etc.}
        \label{fig:asset_price}
    \end{subfigure}
    \hfill
    \begin{subfigure}[b]{0.45\textwidth}
        \centering
        \includegraphics[width=\textwidth]{media/data_preview.pdf}
        \caption{Sample of raw dataset features.}
        \label{fig:data_preview}
    \end{subfigure}
    \caption{Visualization of the dataset showing asset price dynamics and a sample of the input data.}
    \label{fig:data_visuals}
\end{figure}


\section*{Preprocessing}
\subsubsection*{Log Return}
To make the data more stationary and suitable for modelling, we first compute the \textbf{log returns} of each asset. This is to normalize price movements and stabilize variance across different scales. The log return for time $k$ is calculated as  
\begin{align*}
    r_k = \ln\left(\frac{P_k}{P_{k-1}}\right),
\end{align*}
\noindent where $P_k$ represents the asset price at time $k$. It also helps mitigate the effects of large price discrepancies between assets, ensuring that model training focuses on relative changes rather than absolute price levels. Compared to Figure \ref{fig:asset_price}, below shows the log-return transformed of all asset prices.
\begin{figure}[htbp]
    \centering
    \includegraphics[width=0.7\textwidth]{media/all_asset_log_returns.pdf}
    \caption{Log-return transformed asset price trends.}
    \label{fig:all_asset_log_returns}
\end{figure}



% \subsubsection*{Dimension Reduction}

% Here we perform \textbf{lag-principal component analysis (Lag-PCA)} to identify the most informative features in the dataset and reduce dimensionality. Since we are dealing with time-series data, we create a lagged version by sliding a window of size $l = 70$ over the time series. For each data chunk, we perform PCA independently to capture local temporal patterns. We then apply a \textbf{majority voting} scheme across all chunks to determine the most significant principal components ($n = 10$) to retain for model training. The majority voting is updated via the following heuristic:

% \begin{align*}
%     V_i \leftarrow V_i + n * \frac{1}{\text{rank}_i}.
% \end{align*}

% Here, $\text{rank}_i$ denotes the rank of the $i^{\text{th}}$ component within a chunk, and $n$ is the number of components considered. The components with the highest aggregated scores across all windows are selected as the final features for model training. Figure \ref{fig:pca_feature_importance} shows the final result of the Lag-PCA.



\subsubsection*{Dimension Reduction}
Here we perform \textbf{lag-principal component analysis (Lag-PCA)} to identify the most informative features in the dataset and reduce dimensionality. Our core hypothesis is that financial time-series data is \textbf{non-stationary}; the importance of a given feature (e.g., 'Crude\_oil') may change significantly over time. A single, "global" PCA performed on the entire dataset would average out these local temporal dynamics.

To capture these time-varying relationships, we create data chunks by sliding a window of size $l = 70$ over the time series. For each 70-day chunk, we perform PCA independently. This local PCA allows us to rank the importance of the  \textit{original features} based on their contribution (i.e., factor loadings) to the principal components \textit{within that specific time window}.

We then apply a \textbf{majority voting} scheme to aggregate these local rankings and identify the features that are \textit{consistently} important across all chunks. We maintain a global score vector $V$, where $V_i$ is the total score for the $i^{\text{th}}$ original feature. For each window, we update this score using the following heuristic:
\begin{align*}
    V_i \leftarrow V_i + n * \frac{1}{\text{rank}_i}.
\end{align*}
Here, $\text{rank}_i$ denotes the importance rank of the $i^{\text{th}}$ feature (from 1 to 36) within the current chunk, and $n$ is the number of principal components we consider to determine this rank (which we set to $n = 10$). This heuristic strongly rewards features that are frequently ranked as most important (i.e., have a low $\text{rank}_i$).

After sliding the window across the entire dataset, the 10 \textit{original features} with the highest aggregated scores ($V_i$) are selected as the final, most informative features for model training. Figure \ref{fig:pca_feature_importance} shows the final aggregated scores from this Lag-PCA process.




\begin{figure}[htbp]
    \centering
    \includegraphics[width=0.5\textwidth]{media/pca_feature_importance.pdf}
    \caption{Lag-PCA feature importance.}
    \label{fig:pca_feature_importance}
\end{figure}

% \section*{Methods}

% In this project, we implement and compare three learning-based time-series models: Long Short-Term Memory (LSTM) networks, Temporal Convolutional Networks (TCNs), and Transformer-based models. 

% \subsubsection*{LSTM}
% We implement the LSTM model using PyTorch. The architecture consists of two LSTM layers followed by two fully connected layers. The model is trained using the Adam optimizer with a learning rate of 0.001 and Mean Squared Error (MSE) as the loss function. We employ early stopping based on validation loss to prevent overfitting.

% Below shows the model architecture:
% \textbf{include figure here}


% \section*{Preliminary Results}
% Here we include the preliminary results of the LSTM model. The training and validation loss curves are shown in Figure \ref{fig:lstm_loss_curve}. We observe that the training loss decreases steadily, while the validation loss starts to increase after a certain number of epochs, indicating potential overfitting.
\vspace{-24pt}
\section*{Methods}
In this project report, we implement and compare three learning-based time-series models: Long Short-Term Memory (LSTM) networks, Temporal Convolutional Networks (TCNs), and Transformer-based models. All models are trained to take a sequence of past data to predict the S\&P 500 log return for the following day.

\subsubsection*{LSTM}
We implement the LSTM model using PyTorch. The model's architecture is designed to handle sequences of the $N=10$ features selected by our Lag-PCA.

The model takes an input tensor of shape ($W$, $N$), where $W$ is the lookback window (e.g., $W=10$ previous days) and $N=10$ is the number of features. This sequence is processed by a single LSTM layer with $H=32$ hidden units. We use the output from the final time step of the LSTM layer, which represents the encoded state of the entire sequence. This output (of size $H=32$) is then passed through a dropout layer (with dropout rate 0.3) followed by two fully connected (Linear) layers. The first fully connected layer reduces the dimension from $H=32$ to $F=16$, and the final layer maps this to a single output value ($O=1$), representing the predicted log return for the next time step.



The model is trained using the Adam optimizer with a learning rate of 0.001 and Mean Squared Error (MSE) as the loss function. We employ early stopping, monitoring the validation loss with a patience of 10 epochs, to prevent overfitting and save the best-performing model.

\section*{Preliminary Results}
Here we include the preliminary results of the LSTM model. The training and validation loss curves are shown in Figure \ref{fig:lstm_loss_curve}. We observe that the training loss decreases steadily, while the validation loss starts to increase after a certain number of epochs, indicating potential overfitting.

\begin{figure}
    \centering
    \includegraphics[width=0.45\linewidth]{media/training_loss_improved.pdf}
    \caption{Training and Test Loss for the Improved LSTM model (1 LSTM layer, 32 hidden units, 2 FC layers, and Dropout)}
    \label{fig:lstm_loss_curve}
\end{figure}







% \section*{Current Conclusions, Unexpected Challenges, Next Steps}
% \begin{itemize}
%     \item Successfully implemented the \textbf{LSTM} model for time-series prediction.
%     \item Observed that the test loss is larger than the training loss, indicating possible \textbf{overfitting}. To mitigate this, we plan to gather and incorporate more data.
%     \item With sufficient data, we intend to implement and compare the performance of \textbf{Temporal Convolutional Networks (TCN)} and \textbf{Transformer-based models} against the LSTM.
%     \item Additionally, we will compare the results with traditional \textbf{regression models} to evaluate relative performance.
%     \item We plan to investigate the effect of dimensionality reduction by examining whether reducing the number of \textbf{PCA components} significantly degrades model performance.
%     \item Finally, we aim to train a model using the \textbf{raw price data} (without log-return transformation) and compare its results with the log-return–based model.
% \end{itemize}

\section*{Conclusions and Future Work}


\begin{itemize}
    \item The initial \textbf{LSTM} model for time-series prediction has been successfully implemented and evaluated.
    \item We observed that the test loss is notably higher than the training loss, which indicates potential \textbf{overfitting} of the model. To mitigate this, our primary plan is to expand the dataset by incorporating more historical data.
    \item Pending dataset expansion, we intend to implement and benchmark the performance of \textbf{Temporal Convolutional Networks (TCN)} and \textbf{Transformer-based models} against the established LSTM baseline.
    \item Additionally, the performance of these deep learning approaches will be contextualized by comparing their results against traditional \textbf{regression models}.
    \item We also plan to investigate the impact of our dimensionality reduction strategy by conducting an analysis on whether reducing the number of \textbf{PCA components} significantly degrades model performance.
    \item Finally, we aim to conduct a comparative study by training a model on the \textbf{raw price data} (without the log-return transformation) to validate the efficacy of our preprocessing methodology.
\end{itemize}

% \subsubsection*{TCN}

% \vspace{3mm}

\section*{Work Breakdown:}
\begin{itemize}
    \item \textbf{Mohammad Afrazi:} Data preprocessing, feature engineering, writing, and model training/evaluation.
    \item \textbf{Patrick Lo:} Data preprocessing, feature engineering, writing, and model training/evaluation.
\end{itemize}
We, Mohammad Afrazi and Patrick Lo, affirm that we have adhered to the Duke University Honor Code in the completion of this project.


\renewcommand{\bibname}{References}

\bibliographystyle{plain}
\bibliography{Reference}

\end{document}

